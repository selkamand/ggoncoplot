\nonstopmode{}
\documentclass[a4paper]{book}
\usepackage[times,inconsolata,hyper]{Rd}
\usepackage{makeidx}
\usepackage[utf8]{inputenc} % @SET ENCODING@
% \usepackage{graphicx} % @USE GRAPHICX@
\makeindex{}
\begin{document}
\chapter*{}
\begin{center}
{\textbf{\huge Package `ggoncoplot'}}
\par\bigskip{\large \today}
\end{center}
\inputencoding{utf8}
\ifthenelse{\boolean{Rd@use@hyper}}{\hypersetup{pdftitle = {ggoncoplot: Easily Create Interactive Oncoplots}}}{}
\ifthenelse{\boolean{Rd@use@hyper}}{\hypersetup{pdfauthor = {Sam El-Kamand}}}{}
\begin{description}
\raggedright{}
\item[Type]\AsIs{Package}
\item[Title]\AsIs{Easily Create Interactive Oncoplots}
\item[Version]\AsIs{0.1.0}
\item[Description]\AsIs{Generate oncoplots from tabular mutational data. 
Optionally make these oncoplots interactive, with a fully customisable tooltip.}
\item[License]\AsIs{MIT + file LICENSE}
\item[Encoding]\AsIs{UTF-8}
\item[LazyData]\AsIs{true}
\item[RoxygenNote]\AsIs{7.2.0}
\item[Imports]\AsIs{assertthat,
cli,
dplyr,
forcats,
ggiraph,
ggplot2,
RColorBrewer}
\item[URL]\AsIs{}\url{https://github.com/selkamand/ggoncoplot}\AsIs{,
}\url{https://selkamand.github.io/ggoncoplot/}\AsIs{}
\item[BugReports]\AsIs{}\url{https://github.com/selkamand/ggoncoplot/issues}\AsIs{}
\item[Suggests]\AsIs{knitr,
rmarkdown}
\item[VignetteBuilder]\AsIs{knitr}
\end{description}
\Rdcontents{\R{} topics documented:}
\inputencoding{utf8}
\HeaderA{ggoncoplot}{GG oncoplot}{ggoncoplot}
%
\begin{Description}\relax
GG oncoplot
\end{Description}
%
\begin{Usage}
\begin{verbatim}
ggoncoplot(
  .data,
  col_genes,
  col_samples,
  col_mutation_type = NULL,
  col_tooltip = col_samples,
  topn = 10,
  show_sample_ids = FALSE,
  interactive = TRUE,
  interactive_svg_width = 12,
  interactive_svg_height = 6,
  genes_to_include = NULL,
  xlab_title = "Sample",
  ylab_title = "Gene",
  sample_annotation_df = NULL
)
\end{verbatim}
\end{Usage}
%
\begin{Arguments}
\begin{ldescription}
\item[\code{.data}] data for oncoplot. A data.frame with 1 row per mutation in your cohort. Must contain columns describing gene\_symbols and sample\_identifiers, (data.frame)

\item[\code{col\_genes}] name of \strong{data\_main} column containing gene names/symbols (string)

\item[\code{col\_samples}] name of \strong{data\_main} column containing sample identifiers (string)

\item[\code{col\_mutation\_type}] name of \strong{data\_main} column describing mutation types (string)

\item[\code{col\_tooltip}] name of \strong{data\_main} column containing whatever information you want to display in (string)

\item[\code{topn}] how many of the top genes to visualise. Ignored if \code{genes\_to\_include} is supplied (number)

\item[\code{show\_sample\_ids}] show sample\_ids\_on\_x\_axis (flag)

\item[\code{interactive}] should plot be interactive (boolean)

\item[\code{interactive\_svg\_width}] dimensions of interactive plot (number)

\item[\code{interactive\_svg\_height}] dimensions of interactive plot (number)

\item[\code{genes\_to\_include}] specific genes to include in the oncoplot (character)

\item[\code{xlab\_title}] x axis lable (string)

\item[\code{ylab\_title}] y axis of interactive plot (number)

\item[\code{sample\_annotation\_df}] a data.frame with 1 row per sample, with columns representing metadata to annotate oncoplot with. 1st column must contain sample identifiers (data.frame)
\end{ldescription}
\end{Arguments}
%
\begin{Value}
ggplot or ggiraph object if \code{interactive=TRUE}
\end{Value}
%
\begin{Examples}
\begin{ExampleCode}
# ===== GBM =====
gbm_csv <- system.file(
  package = "ggoncoplot",
  "testdata/GBM_tcgamutations_mc3_maf.csv.gz"
)

gbm_df <- read.csv(file = gbm_csv, header = TRUE)

ggoncoplot(
  gbm_df,
  "Hugo_Symbol",
  "Tumor_Sample_Barcode",
  col_mutation_type = "Variant_Classification"
)

\end{ExampleCode}
\end{Examples}
\inputencoding{utf8}
\HeaderA{score\_based\_on\_gene\_rank}{Generate score based on genes}{score.Rul.based.Rul.on.Rul.gene.Rul.rank}
%
\begin{Description}\relax
Score used to sort samples based on which genes are mutated. Make sure to run on one sample at once (use grouping)
\end{Description}
%
\begin{Usage}
\begin{verbatim}
score_based_on_gene_rank(
  mutated_genes,
  genes_informing_score,
  gene_rank,
  debug_mode = FALSE
)
\end{verbatim}
\end{Usage}
%
\begin{Arguments}
\begin{ldescription}
\item[\code{mutated\_genes}] vector of genes that are mutated for a single sample (character)

\item[\code{genes\_informing\_score}] which genes determine the sort order? (character)

\item[\code{gene\_rank}] what is the order of importance of genes used to determine sort order. Higher number = higher in sort order (character)

\item[\code{debug\_mode}] debug mode (flag)
\end{ldescription}
\end{Arguments}
%
\begin{Value}
a score (higher = should be higher in the sorting order) (number)
\end{Value}
%
\begin{Examples}
\begin{ExampleCode}
## Not run: 
# First set of genes has a high rank since both BRCA2 and EGFR are mutated
score_based_on_gene_rank(c("TERT", "EGFR", "PTEN", "BRCA2"), c("EGFR", "BRCA2"), gene_rank = 1:2)

# If EGFR is mutated without BRCA2, we get a lower score
score_based_on_gene_rank(c("TERT", "EGFR", "PTEN", "IDH1"), c("EGFR", "BRCA2"), gene_rank = 1:2)

# If BRCA2 is mutated without EGFR,
# we get a score lower than BRCA2+EGFR but higher than EGFR alone due to higher gene_rank of BRCA2
score_based_on_gene_rank(c("TERT", "IDH1", "PTEN", "BRCA2"), c("EGFR", "BRCA2"), gene_rank = 1:2)

## End(Not run)
\end{ExampleCode}
\end{Examples}
\inputencoding{utf8}
\HeaderA{theme\_oncoplot\_default}{Oncoplot Theme: default}{theme.Rul.oncoplot.Rul.default}
%
\begin{Description}\relax
Oncoplot Theme: default
\end{Description}
%
\begin{Usage}
\begin{verbatim}
theme_oncoplot_default(...)
\end{verbatim}
\end{Usage}
%
\begin{Arguments}
\begin{ldescription}
\item[\code{...}] passed to [ggplot2::theme()] theme
\end{ldescription}
\end{Arguments}
\printindex{}
\end{document}
